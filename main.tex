% !TeX encoding = UTF-8 Unicode
\documentclass[a4paper,12pt]{article}
\usepackage[english]{babel}
\usepackage[english]{isodate}
\usepackage{graphicx}
\usepackage{hyperref}
\usepackage{centernot}
\usepackage{xfrac}
\usepackage{mathtools}
\usepackage{enumitem}
\usepackage{fancyhdr}
\usepackage{lastpage}
\usepackage{xcolor}
\usepackage{changepage}
\usepackage{amsthm}
\usepackage{amsfonts}
\usepackage{amsmath}
\usepackage{amssymb}
\usepackage{soul}
\usepackage{float}
\usepackage{centernot}
\usepackage{tcolorbox}
\usepackage{titling}
\usepackage{caption}
\usepackage{subcaption}
\usepackage{etoolbox}
\usepackage{listings}
\usepackage{biblatex}
\usepackage{csquotes}
\usepackage{titlesec}
\usepackage{tikz}

\addbibresource{bibliography.bib}

\patchcmd{\section}{\bfseries}{\bfseries\boldmath }{}{}
\patchcmd{\subsection}{\bfseries}{\bfseries\boldmath }{}{}
\patchcmd{\subsubsection}{\bfseries}{\bfseries\boldmath }{}{}

\hypersetup{
  colorlinks=True,
  urlcolor=blue,
  citecolor=red,
  menucolor=black,
}

\renewcommand{\UrlFont}{\ttfamily\footnotesize}

\lstdefinelanguage{futhark}
{
  % list of keywords
  morekeywords={
    do,
    else,
    for,
    if,
    in,
    include,
    let,
    loop,
    then,
    type,
    val,
    while,
    with,
    module,
    def,
    entry,
    local,
    open,
    import,
    assert,
    match,
    case,
  },
  sensitive=true, % Keywords are case sensitive.
  morecomment=[l]{--}, % l is for line comment.
  morestring=[b]" % Strings are enclosed in double quotes.
}


\lstset{
    language=futhark,
    showspaces=false,
    showstringspaces=false,
    mathescape=true,
    aboveskip=0pt,
    belowskip=6pt,
    numberstyle=\tiny,
    numbers=left, 
    firstnumber=1,
    numberfirstline=true,
    tabsize=4,
    breaklines=true,
}

\newcommand\LAST{\text{LAST}}
\newcommand\defiff{\mathrel{\stackrel{\makebox[0pt]{\mbox{\normalfont\tiny def}}}{\iff}}}
\newcommand\defeq{\mathrel{\stackrel{\makebox[0pt]{\mbox{\normalfont\tiny def}}}{=}}}
\newcommand\concat{\: \mathrlap{+} \: +}
\newcommand\map{\textbf{map}\ }
\newcommand\reduce{\textbf{reduce}\ }
\theoremstyle{definition}
\newtheorem{definition}{Definition}[section]
\newtheorem{proposition}{Proposition}[section]
\newtheorem{corollary}{Corollary}[section]
\newtheorem*{remark}{Remark}
\newtheorem{lemma}{Lemma}[section]
\newtheorem{theorem}{Theorem}
\newtheorem{algorithm}{Algorithm}[section]
\newtheorem{example}{Example}[section]
\DeclareMathOperator{\sign}{sign}
\newcommand\doubleplus{+\kern-1.3ex+\kern0.8ex}
\newcommand\mdoubleplus{\ensuremath{\mathbin{+\mkern-10mu+}}}

\newcommand{\id}[1]{\ensuremath{\mathit{#1}}}
\newcommand{\kw}[1]{\ensuremath{\mathtt{#1}}}
\newcommand{\Let}{\kw{let}}
\newcommand{\In}{\kw{in}}
\newcommand{\If}{\kw{if}}
\newcommand{\Then}{\kw{then}}
\newcommand{\Else}{\kw{else}}
\newcommand{\Filter}{\kw{filter}}
\newcommand{\Or}{\kw{or}}
\newcommand{\Groupby}{\kw{groupby}}
\newcommand{\Partition}{\kw{partition}}
\newcommand{\Scatter}{\kw{scatter}}
\newcommand{\Hist}{\kw{hist}}
\newcommand{\Map}{\kw{map}}
\newcommand{\Sum}{\kw{sum}}
\newcommand{\Mod}{\kw{mod}}
\newcommand{\Fst}{\kw{fst}}
\newcommand{\Snd}{\kw{snd}}
\newcommand{\Presum}{\kw{presum}}
\newcommand{\True}{\kw{true}}
\newcommand{\False}{\kw{false}}
\newcommand{\Sort}{\kw{sort}}
\newcommand{\Segor}{\kw{segor}}
\newcommand{\Hash}{\kw{hash}}
\newcommand{\Random}{\kw{random}}
\newcommand{\Iota}{\kw{iota}}
\newcommand{\Unzip}{\kw{unzip}}
\newcommand{\Unit}{\mathbf{unit}}
\newcommand{\Int}{\mathbf{int}}
\newcommand{\Bool}{\mathbf{bool}}
\newcommand{\Def}{\kw{def}}
\newcommand{\Rep}{\kw{rep}}
\newcommand{\Zip}{\kw{zip}}

\fancyhf{}
\setlength{\headheight}{14.49998pt}
\pretitle{\vspace{-120pt}\begin{center}}
\posttitle{\par\end{center}\vspace{-80pt}}
\fancyhead[C]{}
\fancyfoot[R]{Page \thepage \hspace{1pt} of \pageref{LastPage}}
\pagestyle{fancy}
\fancypagestyle{firstpage}{%
    \fancyhf{}
    \renewcommand{\headrulewidth}{0pt}
    \fancyfoot[R]{Page \thepage \hspace{1pt} of \pageref{LastPage}}
}
\title{
    {\Large \textsc{University of Copenhagen}} \\[5pt]
    {\large Union-Find} \\[10pt]
    Author: William Henrich Due \\[0pt]
}
\author{}
\date{}

\begin{document}
\maketitle
\thispagestyle{firstpage}

\section{Theory}
\begin{definition}[Reachability]
    A node $v$ is \emph{reachable} from a node $u$ in a directed graph $G = (V,
    E)$ if there exists a sequence of directed edges $e_1, e_2, \ldots, e_m \in
    E$ where $m \geq 1$ and $e_i = (v_{i-1}, v_i)$ for $1 \leq i \leq m$,
    such that $v_0 = u$ and $v_m = v$. We denote this by $u \leadsto v$.
\end{definition}

\begin{definition}[Cycle]
    A cycle in a directed graph $G = (V, E)$ has a cycle if there exists
    $v \in V$ such that $v \leadsto v$.
\end{definition}

\begin{definition}[Forest]
    A forest is a directed graph $F = (V, E)$ where $V$ is a set of vertices and $E
    \subseteq V \times V$ is a set of directed edges such that:
    \begin{enumerate}
        \item There are no cycles $v \centernot\leadsto v$ for all $v \in V$, and
        \item each node has at most one parent i.e. for all $(u, v_1), (u, v_2)
        \in E$ it holds that $v_1 = v_2$.
    \end{enumerate}
\end{definition}

\begin{definition}[Root]
    A node $v \in V$ in a forest $F = (V, E)$ is a root if it has no parent.
    This is defined as the predicate:
    \begin{align*}
        \mathcal{R}_F(v) : v \centernot\leadsto u \text{ for all } u \in V
    \end{align*}
\end{definition}

\begin{definition}[Tree]
    A tree is a forest $T = (V, E)$ where there exists a unique root $r \in V$
    such that $v \leadsto r$ for all $v \in V\backslash
    \{r\}$.
\end{definition}

\begin{proposition}[Forest Root Count]\label{prop:forest-root-count}
    A forest $F = (V, E)$ where $|V| = n$ and $|E| = n - k$ has $k$ roots.
\end{proposition}

\begin{proof}
    Let $F = (V, E)$ be a forest where $|V| = n$ and $|E| = n - k$. By the
    second property of a forest then $n - k$ vertices must have a parent. Since
    there are $n$ vertices in total it follows that there are exactly $k$
    vertices $r_1, r_2, \ldots, r_k \in V$ that has no parent. Hence there are
    exactly $k$ roots in $F$.
\end{proof}

\begin{proposition}[Roots Path Exist]\label{prop:roots-path-exist}
    In a forest $F = (V, E)$ for each element $v \in V$ there exists at least
    one root $r \in V$ such that $\mathcal{R}_F(r)$ and either $v \leadsto r$ or
    $v = r$.
\end{proposition}

\begin{proof}
    Let $F = (V, E)$ be a forest and let $v \in V$ be an arbitrary element in
    $V$. By proposition \ref{prop:forest-root-count} there exists at least one
    root $r \in V$ such that $\mathcal{R}_F(r)$. This can be shown by structural
    induction on a vertex $v \in V$ that any $(v, p) \in E$ then either $p = r$
    or $p$ has a path to some root $r \in V$.
    \begin{itemize}
        \item If $p = r$ then $p$ is a root and $v$ has a path to a root $r$ by
        $(v, r) \in E$.
        \item By induction hypothesis $p$ has a path to a root $r \in V$ such
        that $\mathcal{R}_F(r)$. Since $(v, p) \in E$ it follows that $v
        \leadsto p \leadsto r$ so .
    \end{itemize}
\end{proof}

\begin{proposition}[Forest Edge Limit]\label{prop:forest-edge-limit}
    A forest $F = (V, E)$ where $|V| = n$ and $|E| > n - 1$ is not a forest.
\end{proposition}

\begin{proof}
    Let $F = (V, E)$ be a forest where $|V| = n$ and $|E| > n - 1$. By
    proposition \ref{prop:forest-root-count} a forest with $n - 1$ has exactly
    one root, so it is a tree. By adding one more edge to the tree it must make
    one vertex have two parents. Or since every vertex $v \in V$ has a path to
    the root $r \in V$ it must create a cycle $v \leadsto v$ for some $v \in V$.
    In both cases it contradicts the properties of a forest.
\end{proof}

\begin{definition}[Representative]
    The representative of an element $v \in V$ in a forest $F = (V, E)$ is the
    root $r \in V$ such that there is a path from $v$ to $r$. This is defined as
    the function:
    \begin{align*}
        \rho_F(v) := r \text{ where } r \in V \text{ such that } \mathcal{R}_F(r) \land (v \leadsto r \lor v = r)
    \end{align*}
\end{definition}

\begin{proposition}[Unique Representative]
    In a forest $F = (V, E)$ each element $v \in V$ has a unique
    representative $\rho_F(v)$.
\end{proposition}

\begin{proof}
    Let $F = (V, E)$ be a union-find structure and let $v \in V$ be an arbitrary
    element in $V$. By proposition \ref{prop:roots-path-exist} there exists at
    least one root $r \in V$ such that $\mathcal{R}_F(r)$ and $v \leadsto r$.
    Now assume that there exists another root $r' \in V$ such that $\mathcal{R}_F(r')$
    and $v \leadsto r'$. Since $F$ is a forest it follows by the second property
    of a forest that $r = r'$ hence the representative is unique.
\end{proof}

\begin{definition}[Tree Set]
    The set of vertices of the same tree $\mathcal{E}_F(v)$ in a
    forest $F = (V, E)$ is defined as:
    \begin{align*}
        \mathcal{E}_F(v) := \{u : u \in V \text{ where } \rho_F(u) = \rho_F(v))\}
    \end{align*}
\end{definition}

\begin{definition}[Partition]\label{def:partition}
    The set $P \subseteq \mathbb{P}(S)$ is a partition of a set $S$ if:
    \begin{enumerate}
        \item $a \neq \emptyset$ for all $a \in P$
        \item $a \cap b = \emptyset$ for all $a, b \in P$ where $a
        \neq b$
        \item $\bigcup_{a \in P} a = S$
    \end{enumerate}
\end{definition}

\begin{proposition}[Forest Partition]\label{prop:forest-partition}
    A forest $F = (V, E)$ is a partition of $V$ for the following set:
    \begin{align*}
        \{\mathcal{E}_F(v) : v \in V\}
    \end{align*}
\end{proposition}

\begin{proof}
    Let $F = (V, E)$ be a forest. We will show that the set in the proposition
    is a partition of $V$ by showing that it satisfies the three properties in
    definition \ref{def:partition}.
    \begin{enumerate}
        \item By definition of $\mathcal{E}_F(v)$ it can not be empty since for
        $\mathcal{E}_F(v)$ then $\rho_F(v) = \rho_F(v)$. Hence $\mathcal{E}_F(v)
        \neq \emptyset$ for all $v \in V$.
        \item Let $a$ and $b$ be two arbitrary elements in the set such that $a
        \neq b$. By definition of $a$ and $b$ there exists $v_1, v_2 \in V$ such
        that $a = \{u : u \in V \land \rho_F(u) = \rho_F(v_1)\}$ and $b = \{u :
        u \in V \land \rho_F(u) = \rho_F(v_2)\}$. Since $a \neq b$ it follows
        that $\rho_F(v_1) \neq \rho_F(v_2)$ since otherwise $a = b$, hence $a
        \cap b = \emptyset$.
        \item Let $v$ be an arbitrary element in $V$. By proposition
        \ref{prop:roots-path-exist} there exists a root $r \in V$ such that
        $\mathcal{R}_F(r)$ and $v \leadsto r$ or $v = r$. By definition of the
        representative it follows that $\rho_F(v) = r$. Now let $a = \{u : u \in
        V \land \rho_F(u) = \rho_F(v)\}$. By definition of $a$ it follows that
        $v \in a$. Since $v$ was arbitrary it follows that $\bigcup_{a \in P} a
        = V$.
    \end{enumerate}
\end{proof}

\begin{definition}[Same Tree Relation]
    The relation $\sim_F$ on a forest $F$ is defined as:
    \begin{align*}
        u \sim_F v :\iff u \in \mathcal{E}_F(v)
    \end{align*}
\end{definition}

\begin{corollary}[Same Tree Relation is an Equivalence Relation]
    The relation $\sim_F$ on a forest $F$ is an equivalence relation due to
    $\{\mathcal{E}_F(v) : v \in V\}$ being a partition of $V$. by proposition
    \ref{prop:forest-partition}.
\end{corollary}

\begin{definition}[Forests with Equivalent Tree Sets]
    Two forests $F = (V, E)$ and $F' = (V', E')$ have equivalent tree sets $F
    \cong F'$ if:
    \begin{itemize}
        \item Vertices are the same $V = V'$.
        \item The tree sets are equivalent $\mathcal{E}_{F}(v) =
        \mathcal{E'}_F(v)$ for all $v \in V$.
    \end{itemize}
\end{definition}

\begin{definition}[Tree Union]
    The tree union of two elements $v$ and $u$ for a forest $F = (V, E)$ is such
    that $v \sim_{F'} u$ in a new forest $F' = (V', E')$ and $F'$ satify the
    following properties:
    \begin{enumerate}
        \item $\mathcal{E}_{F'}(v) = \mathcal{E}_{F'}(u) = \mathcal{E}_F(v) \cup \mathcal{E}_F(u)$ and
        \item $\mathcal{E}_{F'}(w) = \mathcal{E}_F(w) \text{ for all } w \in V
        \backslash (\mathcal{E}_F(v) \cup \mathcal{E}_F(u))$.
    \end{enumerate}
\end{definition}

\begin{proposition}[Tree Union]\label{def:tree-union}
    Let forest $F = (V, E)$, $p = \rho_F(u)$ be the representative of $u$ and
    let $q = \rho_F(v)$ be the representative of $v$ where $q \neq p$. Then
    defined $F'$ as:
    \begin{align*}
        F' &:= (V, E \cup \{(q, p)\})
    \end{align*}
    Then $u \sim_{F'} v$ in $F'$ and $F'$ will satisfy the properties of a tree
    union.
\end{proposition}

\begin{proof}
    Let $F = (V, E)$, $p = \rho_F(u)$ be the representative of $u$ and let $q =
    \rho_F(v)$ be the representative of $v$. By definition $q$ will have parent
    $p$ in $F'$ and since $q$ is a root it has no parent then $F'$ is a forest.
    Now for all $w \in \mathcal{E}_F(q)$ it holds that $w \leadsto q$ or $q = w$
    and since $q \leadsto p$ it follows that $w \leadsto p$. Hence $w \in
    \mathcal{E}_{F'}(p)$ for all $w \in \mathcal{E}_F(q)$ and trivially $w \in
    \mathcal{E}_{F'}(p)$ for all $w \in \mathcal{E}_F(p)$ so it follows that
    $\mathcal{E}_{F'}(v) = \mathcal{E}_{F'}(u) = \mathcal{E}_F(v) \cup
    \mathcal{E}_F(u)$. Now let $w \in V \backslash (\mathcal{E}_F(v) \cup
    \mathcal{E}_F(u))$ be an arbitrary element. Since $w \centernot\leadsto p$,
    $w \neq p$, $w \centernot\leadsto q$, and $w \neq q$ it follows that $w$ has
    the same representative in $F'$ as in $F$ hence $\mathcal{E}_{F'}(w) =
    \mathcal{E}_F(w)$.
\end{proof}

\begin{definition}[Conflict-free Set]
    Let $F$ be a forest, $X \subseteq \{(\rho_F(v), \rho_F(u)) : (v, u) \in V
    \times V\}$ be a set of root pairs. Then $X$ is a conflict-free set in $F$
    if $(V, Y)$ is a forest.
\end{definition}

\begin{proposition}[Conflict-free Forest Union]\label{prop:conflict-free-forest-union}
    Let forest $F = (V, E)$ be a forest and let $X \subseteq V \times V$ be a
    conflict-free set in $F$ where $|X| = n$. Then defining the following forests:
    \begin{align*}
        F_0 &:= F \\
        F_{i} &:= (V, E_{i - 1} \cup \{(v_i, u_i)\}) \text{ for } (v_i, u_i) \in X \text{ and } 1 \leq i \leq n
    \end{align*}
    Then $F_n$ is a forest.
\end{proposition}

\begin{proof}
    Let forest $F = (V, E)$ be a forest, $X \subseteq V \times V$ be a
    conflict-free set in $F$. We will show that $F_n$ is a forest by induction
    on $i$.
    \begin{itemize}
        \item Base case: If $i = 0$ then $F_i = F_0 = F$ which is a forest.
        \item Induction hypothesis: Assume that $F_{i - 1}$ is a forest for all
        $1 \leq i < n$.  Let $(v_i, u_i) \in X$, we know that $v_i \neq v_j$ for
        all $(v_j, u_j) \in Y \backslash \{(v_i, u_i)\}$ since otherwise $(V,
        X)$ would not be a forest and $X$ would not be a conflict-free set in
        $F$. So $v_i$ will only have one parent in $F_i$ since it only appears
        once as a child in $(V, X)$. By definition all of the edges in $X$
        consists of roots in $F$, and since $(V, X)$ is a forest there are no
        cycles $y \centernot\leadsto y$ for all $y \in V$ in $F_i$. Hence $F_i$
        is a forest.
    \end{itemize}
    Thus by induction $F_n$ is a forest.
\end{proof}

\begin{proposition}[Conflict-free Set Equivalence]\label{prop:conflict-free-set-equivalence}
    Let forest $F$ be a forest and let $X \subseteq V \times V$ be a
    conflict-free set in $F$ where $|X| = n$. Then defining the following forests:
    \begin{align*}
        F_0 &:= F \\
        F_{i} &:= (V, E_{i - 1} \cup \{(v_i, u_i)\}) \text{ for } (v_i, u_i) \in X \text{ and } 1 \leq i \leq n \\
        G_0 &:= F \\
        G_{j} &:= (V, E_{i - 1} \cup \{(\rho_{G_{j - 1}}(v_j), \rho_{G_{j - 1}}(u_j))\}) \text{ for } (v_j, u_j) \in X \text{ and } 1 \leq j \leq n
    \end{align*}
    Then $F_{n} \cong G_{n}$.
\end{proposition}
\begin{proof}
    Let forest $F$ be a forest, $X \subseteq V \times V$ be a conflict-free set
    in $F$. We will show that $F_n \cong G_n$. We know that for some $(v_i, u_i)
    \in X$ then $v_i \neq y$ for all  $(y, w) \in X \backslash \{(v_i, u_i)\}$
    since otherwise $(V, X)$ would not be a forest and $X$ would not be a
    conflict-free set in $F$. So all edge set unions will only give a root $v_i$
    a new parent $u_i$ once. So $\rho_{F_n}(v_i) = \rho_{F_n}(u_i)$ and
    $\rho_{G_n}(u_i) = \rho_{G_n}(v_i)$ hence $v_i$ remains in the same tree in
    both $F_n$ and $G_n$. Since this holds for all $(v_i, u_i) \in X$ it follows
    that all elements in $V$ remains in the same tree in both $F_n$ and $G_n$.
    Hence $F_n \cong G_n$.
\end{proof}

\begin{algorithm}[Conflict-free Tree Union]
    Let forest $F$ be a tree and let $Z \subseteq V \times V$ be a set of pairs
    of elements in $V$ that will be unioned in parallel. The conflict-free tree
    union algorithm is defined as:
    \begin{align*}
        & \kw{while} \: Z \neq \emptyset \: \kw{do} \\
        & \quad E \leftarrow \pi_2(F) \\
        & \quad \text{split } Z \text{ into } X, Z' \\
        & \quad \kw{parfor} \: (u, v) \in X \: \kw{do} \\
        & \quad \quad E \leftarrow E \cup \{(u, v)\} \\
        & \quad F \leftarrow (V, E) \\
        & \quad Z \leftarrow Z'
    \end{align*}
\end{algorithm}


\begin{definition}[Isolated Vertex]
    An \emph{isolated vertex} in a directed graph $G = (V, E)$ is a $v \in V$
    such that $(u, v) \notin E$ and $(u, v) \notin E$ for some $u \in V$.
\end{definition}
\begin{proposition}[Edge Vertex Cover]
    Let $G = (V, E)$ be a directed graph without isolated vertices then:
    \begin{align*}
        \{v : (v, u) \in E\} \cup \{u : (v, u) \in E\} = V
    \end{align*}
\end{proposition}
\begin{proof}
    Let $G = (V, E)$ without any isolated vertices. Let $v \in V$ be an
    arbitrary element in $V$. Since $v$ is not an isolated vertex it follows
    that there exists some $u \in V$ such that either $(v, u) \in E$ or $(u, v)
    \in E$. In the first case it follows that $v \in \{v : (v, u) \in E\}$ and
    in the second case it follows that $v \in \{u : (v, u) \in E\}$. Since $v$
    was arbitrary it follows that $\{v : (v, u) \in E\} \cup \{u : (v, u) \in
    E\} = V$.
\end{proof}

\begin{proposition}[Ordered Edges Implies Acyclicity]\label{prop:ordered-edges-implies-acyclicity}
    Let $G = (V, E)$ be a directed graph where for all $(v, u) \in E$ it holds
    that $v < u$ for some total order $(V, <)$. Then $G$ has no cycles.
\end{proposition}

\begin{proof}
    Let $G = (V, E)$ be a directed graph where for all $(u, v) \in E$ it holds
    that $u < v$ for some total order $(V, <)$. Let edges $e_1, e_2, \ldots, e_m
    \in E$ where $m \geq 1$ and $e_i = (v_{i-1}, v_i)$ for $1 \leq i \leq m$ be
    some path in $G$. Since the edges are ordered it follows that:
    \begin{align*}
        v_0 < v_1 < v_2 < \cdots < v_{m - 1} < v_m
    \end{align*}
    Hence by transitivity of the total order it follows that $v_0 < v_m$. So
    $v_0 \neq v_m$ hence there are no cycles in $G$.
\end{proof}

\begin{proposition}[Inverted Acyclic Graph is Acyclic]\label{prop:inverted-acyclic-graph}
    Let $G = (V, E)$ be a directed acyclic graph. Then the inverted graph $G' =
    (V, E')$ where $E' = \{(u, v) : (v, u) \in E\}$ is also acyclic.
\end{proposition}
\begin{proof}
    Let $G = (V, E)$ be a directed acyclic graph and $G' = (V, E')$ where $E' =
    \{(u, v) : (v, u) \in E\}$ is the inverted graph. Let edges $e_1, e_2,
    \ldots, e_m \in E'$ where $m \geq 1$ and $e_i = (v_{i-1}, v_i)$ for $1 \leq
    i \leq m$ be some path in $G'$. By definition of $E'$ it follows that there
    exists edges $e'_1, e'_2, \ldots, e'_m \in E$ where $e'_i = (v_i, v_{i-1})$
    for $1 \leq i \leq m$. If there was a cycle in $G'$ then it would hold that
    $v_0 = v_m$. But since $G$ is acyclic it follows that $v_0 \neq v_m$. Hence
    there are no cycles in $G'$.  
\end{proof}

\begin{algorithm}[Left Maximal Union]\label{alg:left-maximal-union}
    Let forest $F = (V, E)$ be a forest and let $Z \subseteq \{(\rho_F(v),
    \rho_F(u)) : (v, u) \in V \times V\}$ be a set of root pairs $F$ and $(V,
    Z)$ is an acyclic directed graph. The left maximal conflict-free set
    algorithm is defined as:
    \begin{align*}
        & \id{LeftMaximalUnion}(F, Z) \\ 
        1. & \qquad (V, E) \leftarrow F \\
        2. & \qquad \text{Let } X \subseteq Z \text{ where } \{v : (v, u) \in X\} = \{v : (v, u) \in Z\} \\
        & \qquad \qquad \qquad \quad \: \: \: 
        \: \text{ and } |X| = |\{v : (v, u) \in Z\}| \\
        3. & \qquad E \leftarrow E \cup X \\
        4. & \qquad \kw{return} \: ((V, E), Z \backslash X)
    \end{align*}
\end{algorithm}
\begin{proposition}[Left Maximal Union Correctness]\label{prop:left-maximal-union-correctness}
    Let forest $F = (V, E)$ be a forest and let $Z \subseteq \{(\rho_F(v),
    \rho_F(u)) : (v, u) \in V \times V\}$ be a set of root pairs $F$ and $(V,
    Z)$ is an acyclic directed graph. Then the left maximal conflict-free set
    algorithm returns a forest $F' = (V, E')$ and a conflict-free set $X
    \subseteq Z$ in $F$.
\end{proposition}
\begin{proof}
    Let forest $F = (V, E)$ be a forest and let $Z \subseteq \{(\rho_F(v),
    \rho_F(u)) : (v, u) \in V \times V\}$ be a set of root pairs $F$ and $(V,
    Z)$ is an acyclic directed graph. Since $X$ is defined such that $\{v : (v,
    u) \in X\} = \{v : (v, u) \in Z\}$ and $|X| = |\{v : (v, u) \in Z\}|$ it
    follows that $X$ is a conflict-free set in $F$ since no vertex $v$ appears
    more than once as a child in $X$ i.e. $(V, X)$ is a forest. By adding the
    edges in $X$ to $E$ it follows by proposition
    \ref{prop:conflict-free-forest-union} that $F' = (V, E')$ is a forest where
    $E' = E \cup X$.
\end{proof}

\begin{proposition}[Left Maximal Union Time Complexity]
    Let forest $F = (V, E)$ be a forest and let $Z \subseteq \{(\rho_F(v),
    \rho_F(u)) : (v, u) \in V \times V\}$ be a set of root pairs $F$ and $(V,
    Z)$ is an acyclic directed graph. Then the left maximal conflict-free set
    algorithm runs in $O(|Z|)$ work and $O(\log |Z|)$ depth.
\end{proposition}

\begin{proof}
    For step 1. it takes $O(1)$ work and $O(1)$ if we assume that $E$ is only
    used once in this function. Step 2. can be implemented by a parallel sort on
    the first element of each pair in $Z$ followed by a parallel filter that
    selects the first occurrence of each unique first element. This takes
    $O(|Z|)$ work using radix sort with a fixed key length and $O(\log |Z|)$
    depth. Step 3. takes $O(|X|)$ work and $O(1)$ depth to add the edges in $X$
    to $E$. Step 4. takes $O(|Z|)$ work and $O(\log |Z|)$ depth to compute the
    set difference $Z \backslash X$ by a filter. Hence the total work is
    $O(|Z|)$ and the total depth is $O(\log |Z|)$. Hence the left maximal
    conflict-free set algorithm runs in $O(|Z|)$ work and $O(\log |Z|)$ depth.
\end{proof}


\begin{proposition}[Acyclic Directed Graph Presevation]\label{prop:acyclic-directed-graph-preservation}
    Let $G = (V, E)$ be a directed acyclic graph and let $F = (V, X)$ be a
    forest where $X \subseteq E$. Then the directed graph $G' = (V, E')$ where
    $E' = \{(\rho_F(v), \rho_F(u)) : (v, u) \in E \backslash X \land \rho_F(v)
    \neq \rho_F(u)\}$ is also acyclic.
\end{proposition}
\begin{proof}
    Let $G = (V, E)$ be a directed acyclic graph and let $F = (V, X)$ be a
    forest where $X \subseteq E$. Let $q$ be a path $e_1, e_2, \ldots, e_m \in
    E$ where $m \geq 1$ and $e_i = (v_{i-1}, v_i)$ for $1 \leq i \leq m$ be some
    path in $G$.
    \begin{itemize}
        \item If all edges $e_i \notin X$ then the path $q$ exists in $G'$ and
        does not form a cycle since $G$ is acyclic.
        \item If $q$ has some subpath $e_j, e_{j+1}, \ldots, e_k \in X$ for $1
        \leq j < k \leq m$ where $(v_{j - 1}, v_j) \in X$ then $\rho_F(v_{j-1})
        = \rho_F(v_{j})$ so it will not be in $E \backslash X$. Let $(v, u) \in
        E$ where $u  = v_{j - 1}$ or $u = v_j$ then $\rho_F(u) = v_k$ so the
        edge in $G'$ will be $(\rho_F(v), v_k)$.
        \begin{itemize}
            \item If $v \in \{w : (w, u) \in X\}$ then $\rho_F(v) = v_k$ so the
            edge is $(v_k, v_k)$ which is a trivial cycle and since $\rho_F(v)
            \neq \rho_F(u)$ it will not be in $E'$.
            \item If $v \notin \{w : (w, u) \in X\}$ then $(\rho_F(v), v_k)$
            will be an edge in $G'$. And since the path $v \leadsto v_k$ in $G$
            does not form a cycle it follows that single edge from $(\rho_F(v),
            v_k) \in E'$ does not form a cycle either.
        \end{itemize}
    \end{itemize}
\end{proof}

\begin{algorithm}[Parallel Tree Union]
    Let forest $F = (V, E)$ be a tree and let $A \subseteq V \times V$ be a set of pairs
    of elements in $V$ that will be unioned in parallel. The parallel tree union
    algorithm is defined as:
    \begin{align*}
        & \id{ParallelTreeUnion}(F, A) \\
        1. & \qquad Z_p \leftarrow \{(\rho_F(v), \rho_F(u)) : (v, u) \in A \land \rho_F(v) \neq \rho_F(u)\} \\
        2. & \qquad Z_o \leftarrow \{(\min \{v, u\}, \max \{v, u\}) : (v, u) \in Z'\} \\
        3. & \qquad (F_1, Z_1) \leftarrow \id{LeftMaximalUnion}(F, Z_o) \\
        4. & \qquad Z_q \leftarrow \{(\rho_{F_1}(v), \rho_{F_1}(u)) : (v, u) \in Z_1 \land \rho_{F_1}(v) \neq \rho_{F_1}(u)\} \\
        5. & \qquad Z_s \leftarrow \{(u, v) : (v, u) \in Z_q\} \\
        6. & \qquad (F_2, Z_2) \leftarrow \id{LeftMaximalUnion}(F_1, Z_s) \\
        7. & \qquad \kw{return} \: F_2
    \end{align*}
\end{algorithm}

\begin{proposition}[Parallel Tree Union Correctness]
    Let forest $F = (V, E)$ be a tree and let $A \subseteq V \times V$ be a set of pairs
    of elements in $V$ that will be unioned in parallel. Then the parallel tree union
    algorithm returns a forest $F' = (V, E')$ where for all $(v, u) \in A$ it holds that
    $v \sim_{F'} u$.
\end{proposition}

\begin{proof}
    Let forest $F = (V, E)$ be a tree and let $A \subseteq V \times V$ be a set
    of pairs. At step 1. by definition of $Z_p$ it holds that for all $(v, u)
    \in A$ then $(\rho_F(v), \rho_F(u)) \in Z_p$ where $\rho_F(v) \neq
    \rho_F(u)$. In step 2. by proposition
    \ref{prop:ordered-edges-implies-acyclicity} it follows that $(V, Z_o)$ is an
    acyclic directed graph since for all $(v, u) \in Z_o$ it holds that $v < u$.
    So in step 3. left maximal union can be performed and by proposition
    \ref{prop:left-maximal-union-correctness} it follows that $F_1$ is a forest
    and $Z_1$ is the remaining elements in $Z_o$ that was not added to $F_1$. In
    step 4. by proposition \ref{prop:acyclic-directed-graph-preservation} it
    follows that $(V, Z_q)$ is an acyclic directed graph. In step 5. by
    proposition \ref{prop:inverted-acyclic-graph} it follows that $(V, Z_s)$ is
    also an acyclic directed graph. So in step 6. left maximal union can be
    performed again. Hence $F_2$ is a forest being returned in step 7. and it
    fulfills that it has unioned some of the pairs in $A$.

    It now remains to show that for all $(v, u) \in A$ where $\rho_F(v) \neq
    \rho_F(u)$ that every pair is unioned in $F_2$.
    \begin{itemize}
        \item If $(\rho_F(v), \rho_F(u))$ was added to $F_1$ in step 3. then by
         definition of left maximal union all $\rho_F(v) \in \{w : (w, y) \in
         A\}$ will be given some parent $w \in V$ and since $F_2$ is formed by
         adding more edges to $F_1$ it follows that $v$ has the same parent in
         $F_2$ as in $F_1$.
        \item If $(\rho_F(v), \rho_F(u))$ was not added to $F_1$ in step 3. then
        it is in $Z_1$. In step 6. we have the edge has become
        $(\rho_{F_1}(\rho_{F}(u)), \rho_{F_1}(\rho_{F}(v))) \in Z_s$ where
        $\rho_{F_1}(\rho_F(u)) \neq \rho_{F_1}(\rho_{F}(v))$. If $u \notin \{w :
        (w, y) \in A\}$ then $\rho_{F_1}(\rho_{F}(u)) = \rho_{F}(u)$ since it
        has not been given a parent in step 3. So remaining $u \in \{y : (w, y)
        \in A\}$ will be given a parent in step 6.
    \end{itemize}
    Hence $\{w : (w, y) \in A\} \cup \{y : (w, y) \in A\}$ covers all elements
    to be unioned in $A$ and since all these elements are given some parent in
    either step 3. or step 6. if they do not introduce cycles, it follows that
    for all $(v, u) \in A$ it holds that $v \sim_{F'} u$ in $F' = F_2$.
\end{proof}

\begin{definition}[Union-Find Structure]
    The union-find structure $U$ is a forest $U = (V, E)$ where the vertices $V
    = S$ for some set of elements $S$. The edges $E \subseteq V \times V$
    represent parent relations between elements in $S$ such that $(u, v) \in E$
    means that $u$ has parent $v$.
\end{definition}



\end{document}
