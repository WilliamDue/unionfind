% !TeX encoding = UTF-8 Unicode
\documentclass[a4paper,12pt]{article}
\usepackage[english]{babel}
\usepackage[english]{isodate}
\usepackage{graphicx}
\usepackage{hyperref}
\usepackage{centernot}
\usepackage{xfrac}
\usepackage{mathtools}
\usepackage{enumitem}
\usepackage{fancyhdr}
\usepackage{lastpage}
\usepackage{xcolor}
\usepackage{changepage}
\usepackage{amsthm}
\usepackage{amsfonts}
\usepackage{amsmath}
\usepackage{amssymb}
\usepackage{soul}
\usepackage{float}
\usepackage{centernot}
\usepackage{tcolorbox}
\usepackage{titling}
\usepackage{caption}
\usepackage{subcaption}
\usepackage{etoolbox}
\usepackage{listings}
\usepackage{biblatex}
\usepackage{csquotes}
\usepackage{titlesec}
\usepackage{tikz}

\addbibresource{bibliography.bib}

\patchcmd{\section}{\bfseries}{\bfseries\boldmath }{}{}
\patchcmd{\subsection}{\bfseries}{\bfseries\boldmath }{}{}
\patchcmd{\subsubsection}{\bfseries}{\bfseries\boldmath }{}{}

\hypersetup{
  colorlinks=True,
  urlcolor=blue,
  citecolor=red,
  menucolor=black,
}

\renewcommand{\UrlFont}{\ttfamily\footnotesize}

\lstdefinelanguage{futhark}
{
  % list of keywords
  morekeywords={
    do,
    else,
    for,
    if,
    in,
    include,
    let,
    loop,
    then,
    type,
    val,
    while,
    with,
    module,
    def,
    entry,
    local,
    open,
    import,
    assert,
    match,
    case,
  },
  sensitive=true, % Keywords are case sensitive.
  morecomment=[l]{--}, % l is for line comment.
  morestring=[b]" % Strings are enclosed in double quotes.
}


\lstset{
    language=futhark,
    showspaces=false,
    showstringspaces=false,
    mathescape=true,
    aboveskip=0pt,
    belowskip=6pt,
    numberstyle=\tiny,
    numbers=left, 
    firstnumber=1,
    numberfirstline=true,
    tabsize=4,
    breaklines=true,
}

\newcommand\LAST{\text{LAST}}
\newcommand\defiff{\mathrel{\stackrel{\makebox[0pt]{\mbox{\normalfont\tiny def}}}{\iff}}}
\newcommand\defeq{\mathrel{\stackrel{\makebox[0pt]{\mbox{\normalfont\tiny def}}}{=}}}
\newcommand\concat{\: \mathrlap{+} \: +}
\newcommand\map{\textbf{map}\ }
\newcommand\reduce{\textbf{reduce}\ }
\theoremstyle{definition}
\newtheorem{definition}{Definition}[section]
\newtheorem{proposition}{Proposition}[section]
\newtheorem{corollary}{Corollary}[section]
\newtheorem*{remark}{Remark}
\newtheorem{lemma}{Lemma}[section]
\newtheorem{theorem}{Theorem}
\newtheorem{algorithm}{Algorithm}[section]
\newtheorem{example}{Example}[section]
\DeclareMathOperator{\sign}{sign}
\newcommand\doubleplus{+\kern-1.3ex+\kern0.8ex}
\newcommand\mdoubleplus{\ensuremath{\mathbin{+\mkern-10mu+}}}

\newcommand{\id}[1]{\ensuremath{\mathit{#1}}}
\newcommand{\kw}[1]{\ensuremath{\mathtt{#1}}}
\newcommand{\Let}{\kw{let}}
\newcommand{\In}{\kw{in}}
\newcommand{\If}{\kw{if}}
\newcommand{\Then}{\kw{then}}
\newcommand{\Else}{\kw{else}}
\newcommand{\Filter}{\kw{filter}}
\newcommand{\Or}{\kw{or}}
\newcommand{\Groupby}{\kw{groupby}}
\newcommand{\Partition}{\kw{partition}}
\newcommand{\Scatter}{\kw{scatter}}
\newcommand{\Hist}{\kw{hist}}
\newcommand{\Map}{\kw{map}}
\newcommand{\Sum}{\kw{sum}}
\newcommand{\Mod}{\kw{mod}}
\newcommand{\Fst}{\kw{fst}}
\newcommand{\Snd}{\kw{snd}}
\newcommand{\Presum}{\kw{presum}}
\newcommand{\True}{\kw{true}}
\newcommand{\False}{\kw{false}}
\newcommand{\Sort}{\kw{sort}}
\newcommand{\Segor}{\kw{segor}}
\newcommand{\Hash}{\kw{hash}}
\newcommand{\Random}{\kw{random}}
\newcommand{\Iota}{\kw{iota}}
\newcommand{\Unzip}{\kw{unzip}}
\newcommand{\Unit}{\mathbf{unit}}
\newcommand{\Int}{\mathbf{int}}
\newcommand{\Bool}{\mathbf{bool}}
\newcommand{\Def}{\kw{def}}
\newcommand{\Rep}{\kw{rep}}
\newcommand{\Zip}{\kw{zip}}

\fancyhf{}
\setlength{\headheight}{14.49998pt}
\pretitle{\vspace{-120pt}\begin{center}}
\posttitle{\par\end{center}\vspace{-80pt}}
\fancyhead[C]{}
\fancyfoot[R]{Page \thepage \hspace{1pt} of \pageref{LastPage}}
\pagestyle{fancy}
\fancypagestyle{firstpage}{%
    \fancyhf{}
    \renewcommand{\headrulewidth}{0pt}
    \fancyfoot[R]{Page \thepage \hspace{1pt} of \pageref{LastPage}}
}
\title{
    {\Large \textsc{University of Copenhagen}} \\[5pt]
    {\large Applied High Performance Computing} \\[10pt]
    Author: William Henrich Due \\[0pt]
}
\author{}
\date{}

\begin{document}
\maketitle
\thispagestyle{firstpage}

\section{Definitions}

\begin{definition}[Union-Find]\label{def:union-find} The
    union-find data structure $D$ represents a partition of a set
    $S$ if:
    \begin{enumerate}
        \item $a \neq \emptyset$ for all $a \in \mathcal{C}(D)$
        \item $a \cap b = \emptyset$ for all $a, b \in \mathcal{C}(D)$ where $a \neq b$
        \item $\bigcup_{a \in \mathcal{C}(D)} a = S$
    \end{enumerate}
    where $\mathcal{C} : \mathbb{P}(D) \to \mathbb{P}(\mathbb{P}(S))$ converts the data structure $D$ into a set of sets of positive integers.
\end{definition}

\begin{definition}{Union-find set}
    The set $\mathbb{U}_n = \mathbb{P}(\mathbb{N}_{\leq n} \times (\mathbb{N}_{\leq n} \cup \{0\}) \times S)$ is the set of all union-find data structures of size $n$. Each element $U \in \mathbb{U}_n$ is a set of tuples $(i, p, s)$ where:
    \begin{enumerate}
        \item $i \in \mathbb{N}_{\leq n}$ is a unique index identifying the element.
        \item $p \in \mathbb{N}_{\leq n} \cup \{0\}$ is the index of the parent element. If $p = 0$ then the element is a root.
        \item $s \in S$ is the value associated with the element.
    \end{enumerate}
\end{definition}

\begin{definition}[Cycle]\label{def:cycle}
    A cycle exists in a set $U \in \mathbb{U}_n$ if:
    \begin{enumerate}
        \item $i = p$ for some $(i, p, s) \in U$ or
        \item $p_1 = i_2, p_2 = i_3, \ldots, p_m = i_1$ for some $(i_1, p_1, s_1), (i_2, p_2, s_2), \ldots, (i_m, p_m, s_m) \in U$ where $m > 1$.
    \end{enumerate}
\end{definition}

\begin{definition}[Find Root]
    The root of an element $i$ in a cycleless union-find data structure $U \in \mathbb{U}_n$ is defined as:
    \begin{align*}
        \mathcal{P}_U(i) = 
        \begin{cases}
            i & \text{if } p = 0 \\
            \mathcal{P}_U(p) & \text{if } p \neq 0
        \end{cases} \text{ where } (j, p, s) \in U \land i = j.
    \end{align*}
\end{definition}

\begin{proposition}[Termination of Find Root]
    The function $\mathcal{P}_U(i)$ defined in definition \ref{def:cycle} terminates for all $i$ where $1 \leq i \leq n$ if $U \in \mathbb{U}_n$ has no cycles.
\end{proposition}
\begin{proof}
    The proof is trivially true.
\end{proof}

\begin{proposition}[Representation of Union-Find]\label{def:representation-union-find}
    The union-find structure $U \in \mathbb{U}_n$ fulfills definition \ref{def:union-find} if:
    \begin{enumerate}
        \item Unique indices i.e. $|U| = |A|$ where $A = \{\pi_1(t) : t \in U\}$.
        \item $U$ has no cycles.
        \item The following conversion function:
        \begin{align*}
            \mathcal{C}(D) := \{\{\pi_3(t) : t \in U \land \mathcal{P}_U(\pi_1(t)) = p \} : p \in \mathbb{N}_{\leq n}\} \backslash \{\emptyset\}
        \end{align*}
    \end{enumerate}
\end{proposition}
\begin{proof}
    Let $U \in \mathbb{U}_n$ be a union-find data structure fulfilling the three criteria in proposition \ref{def:representation-union-find}. We will show that $U$ fulfills the three criteria in definition \ref{def:union-find}.

    \begin{enumerate}
        \item By definition of $\mathcal{C}$ it is clear that $a \neq \emptyset$ for all $a \in \mathcal{C}(D)$ since $\mathcal{C}$ only includes non-empty sets.
        \item Let $a, b \in \mathcal{C}(D)$ where $a \neq b$. 
    \end{enumerate}
\end{proof}

\end{document}
