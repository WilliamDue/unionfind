% !TeX encoding = UTF-8 Unicode
\documentclass[a4paper,12pt]{article}
\usepackage[english]{babel}
\usepackage[english]{isodate}
\usepackage{graphicx}
\usepackage{hyperref}
\usepackage{centernot}
\usepackage{xfrac}
\usepackage{mathtools}
\usepackage{enumitem}
\usepackage{fancyhdr}
\usepackage{lastpage}
\usepackage{xcolor}
\usepackage{changepage}
\usepackage{amsthm}
\usepackage{amsfonts}
\usepackage{amsmath}
\usepackage{amssymb}
\usepackage{soul}
\usepackage{float}
\usepackage{centernot}
\usepackage{tcolorbox}
\usepackage{titling}
\usepackage{caption}
\usepackage{subcaption}
\usepackage{etoolbox}
\usepackage{listings}
\usepackage{biblatex}
\usepackage{csquotes}
\usepackage{titlesec}
\usepackage{tikz}

\addbibresource{bibliography.bib}

\patchcmd{\section}{\bfseries}{\bfseries\boldmath }{}{}
\patchcmd{\subsection}{\bfseries}{\bfseries\boldmath }{}{}
\patchcmd{\subsubsection}{\bfseries}{\bfseries\boldmath }{}{}

\hypersetup{
  colorlinks=True,
  urlcolor=blue,
  citecolor=red,
  menucolor=black,
}

\renewcommand{\UrlFont}{\ttfamily\footnotesize}

\lstdefinelanguage{futhark}
{
  % list of keywords
  morekeywords={
    do,
    else,
    for,
    if,
    in,
    include,
    let,
    loop,
    then,
    type,
    val,
    while,
    with,
    module,
    def,
    entry,
    local,
    open,
    import,
    assert,
    match,
    case,
  },
  sensitive=true, % Keywords are case sensitive.
  morecomment=[l]{--}, % l is for line comment.
  morestring=[b]" % Strings are enclosed in double quotes.
}


\lstset{
    language=futhark,
    showspaces=false,
    showstringspaces=false,
    mathescape=true,
    aboveskip=0pt,
    belowskip=6pt,
    numberstyle=\tiny,
    numbers=left, 
    firstnumber=1,
    numberfirstline=true,
    tabsize=4,
    breaklines=true,
}

\newcommand\LAST{\text{LAST}}
\newcommand\defiff{\mathrel{\stackrel{\makebox[0pt]{\mbox{\normalfont\tiny def}}}{\iff}}}
\newcommand\defeq{\mathrel{\stackrel{\makebox[0pt]{\mbox{\normalfont\tiny def}}}{=}}}
\newcommand\concat{\: \mathrlap{+} \: +}
\newcommand\map{\textbf{map}\ }
\newcommand\reduce{\textbf{reduce}\ }
\theoremstyle{definition}
\newtheorem{definition}{Definition}[section]
\newtheorem{proposition}{Proposition}[section]
\newtheorem{corollary}{Corollary}[section]
\newtheorem*{remark}{Remark}
\newtheorem{lemma}{Lemma}[section]
\newtheorem{theorem}{Theorem}
\newtheorem{algorithm}{Algorithm}[section]
\newtheorem{example}{Example}[section]
\DeclareMathOperator{\sign}{sign}
\newcommand\doubleplus{+\kern-1.3ex+\kern0.8ex}
\newcommand\mdoubleplus{\ensuremath{\mathbin{+\mkern-10mu+}}}

\newcommand{\id}[1]{\ensuremath{\mathit{#1}}}
\newcommand{\kw}[1]{\ensuremath{\mathtt{#1}}}
\newcommand{\Let}{\kw{let}}
\newcommand{\In}{\kw{in}}
\newcommand{\If}{\kw{if}}
\newcommand{\Then}{\kw{then}}
\newcommand{\Else}{\kw{else}}
\newcommand{\Filter}{\kw{filter}}
\newcommand{\Or}{\kw{or}}
\newcommand{\Groupby}{\kw{groupby}}
\newcommand{\Partition}{\kw{partition}}
\newcommand{\Scatter}{\kw{scatter}}
\newcommand{\Hist}{\kw{hist}}
\newcommand{\Map}{\kw{map}}
\newcommand{\Sum}{\kw{sum}}
\newcommand{\Mod}{\kw{mod}}
\newcommand{\Fst}{\kw{fst}}
\newcommand{\Snd}{\kw{snd}}
\newcommand{\Presum}{\kw{presum}}
\newcommand{\True}{\kw{true}}
\newcommand{\False}{\kw{false}}
\newcommand{\Sort}{\kw{sort}}
\newcommand{\Segor}{\kw{segor}}
\newcommand{\Hash}{\kw{hash}}
\newcommand{\Random}{\kw{random}}
\newcommand{\Iota}{\kw{iota}}
\newcommand{\Unzip}{\kw{unzip}}
\newcommand{\Unit}{\mathbf{unit}}
\newcommand{\Int}{\mathbf{int}}
\newcommand{\Bool}{\mathbf{bool}}
\newcommand{\Def}{\kw{def}}
\newcommand{\Rep}{\kw{rep}}
\newcommand{\Zip}{\kw{zip}}

\fancyhf{}
\setlength{\headheight}{14.49998pt}
\pretitle{\vspace{-120pt}\begin{center}}
\posttitle{\par\end{center}\vspace{-80pt}}
\fancyhead[C]{}
\fancyfoot[R]{Page \thepage \hspace{1pt} of \pageref{LastPage}}
\pagestyle{fancy}
\fancypagestyle{firstpage}{%
    \fancyhf{}
    \renewcommand{\headrulewidth}{0pt}
    \fancyfoot[R]{Page \thepage \hspace{1pt} of \pageref{LastPage}}
}
\title{
    {\Large \textsc{University of Copenhagen}} \\[5pt]
    {\large Union-Find} \\[10pt]
    Author: William Henrich Due \\[0pt]
}
\author{}
\date{}

\begin{document}
\maketitle
\thispagestyle{firstpage}

\section{Theory}
\begin{definition}[Reachability]
    A node $v$ is \emph{reachable} from a node $u$ in a directed graph $G = (V,
    E)$ if there exists a sequence of directed edges $e_1, e_2, \ldots, e_m \in
    E$ where $m \geq 1$ and $e_i = (v_{i-1}, v_i)$ for $1 \leq i \leq m$,
    such that $v_0 = u$ and $v_m = v$. We denote this by $u \leadsto v$.
\end{definition}

\begin{definition}[Cycle]
    A cycle in a directed graph $G = (V, E)$ has a cycle if there exists
    $v \in V$ such that $v \leadsto v$.
\end{definition}

\begin{definition}[Forest]
    A forest is a directed graph $F = (V, E)$ where $V$ is a set of vertices and $E
    \subseteq V \times V$ is a set of directed edges such that:
    \begin{enumerate}
        \item There are no cycles $v \centernot\leadsto v$ for all $v \in V$, and
        \item each node has at most one parent i.e. for all $(u, v_1), (u, v_2)
        \in E$ it holds that $v_1 = v_2$.
    \end{enumerate}
\end{definition}

\begin{definition}[Root]
    A node $v \in V$ in a forest $F = (V, E)$ is a root if it has no parent.
    This is defined as the predicate:
    \begin{align*}
        \mathcal{R}_F(v) : v \centernot\leadsto u \text{ for all } u \in V
    \end{align*}
\end{definition}

\begin{definition}[Tree]
    A tree is a forest $T = (V, E)$ where there exists a unique root $r \in V$
    such that $v \leadsto r$ for all $v \in V\backslash
    \{r\}$.
\end{definition}

\begin{proposition}[Forest Root Count]\label{prop:forest-root-count}
    A forest $F = (V, E)$ where $|V| = n$ and $|E| = n - k$ has $k$ roots.
\end{proposition}

\begin{proof}
    Let $F = (V, E)$ be a forest where $|V| = n$ and $|E| = n - k$. By the
    second property of a forest then $n - k$ vertices must have a parent. Since
    there are $n$ vertices in total it follows that there are exactly $k$
    vertices $r_1, r_2, \ldots, r_k \in V$ that has no parent. Hence there are
    exactly $k$ roots in $F$.
\end{proof}

\begin{proposition}[Roots Path Exist]\label{prop:roots-path-exist}
    In a forest $F = (V, E)$ for each element $v \in V$ there exists at least
    one root $r \in V$ such that $\mathcal{R}_F(r)$ and $v \leadsto r$.
\end{proposition}

\begin{proof}
    Let $F = (V, E)$ be a forest and let $v \in V$ be an arbitrary element in
    $V$. By proposition \ref{prop:forest-root-count} there exists at least one
    root $r \in V$ such that $\mathcal{R}_F(r)$. This can be shown by structural
    induction on a vertex $v \in V$ that any $p \in V$ such that $v \leadsto p$
    has a path $p \leadsto r$.
    \begin{itemize}
        \item If $v$ is a root then we are done.
        \item By induction hypothesis $v$ has a path to a root $r \in V$ such
        that $\mathcal{R}_F(r)$. Since $v$ is not a root it must have a parent
        $p \in V$ such that $(v, p) \in E$. Since $v \leadsto r$, $v \leadsto p$
        and $v$ only has one parent it follows that $p \leadsto r$. Hence $p
        \leadsto r$ for some root $r \in V$ such that $\mathcal{R}_F(r)$.
    \end{itemize}
\end{proof}

\begin{proposition}[Forest Edge Limit]\label{prop:forest-edge-limit}
    A forest $F = (V, E)$ where $|V| = n$ and $|E| > n - 1$ is not a forest.
\end{proposition}

\begin{proof}
    Let $F = (V, E)$ be a forest where $|V| = n$ and $|E| > n - 1$. By
    proposition \ref{prop:forest-root-count} a forest with $n - 1$ has exactly
    one root, so it is a tree. By adding one more edge to the tree it must make
    one vertex have two parents. Or since every vertex $v \in V$ has a path to
    the root $r \in V$ it must create a cycle $v \leadsto v$ for some $v \in V$.
    In both cases it contradicts the properties of a forest.
\end{proof}

\begin{definition}[Representative]
    The representative of an element $v \in V$ in a forest $F = (V, E)$ is the
    root $r \in V$ such that there is a path from $v$ to $r$. This is defined as
    the function:
    \begin{align*}
        \rho_F(v) := r \text{ where } r \in V \land \mathcal{R}_F(r) \land (v \leadsto r \lor v = r)
    \end{align*}
\end{definition}

\begin{proposition}[Unique Representative]
    In a forest $F = (V, E)$ each element $v \in V$ has a unique
    representative $\rho_F(v)$.
\end{proposition}

\begin{proof}
    Let $U = (V, E)$ be a union-find structure and let $v \in V$ be an arbitrary
    element in $V$. By proposition \ref{prop:roots-path-exist} there exists at
    least one root $r \in V$ such that $\mathcal{R}_U(r)$ and $v \leadsto r$.
    Now assume that there exists another root $r' \in V$ such that $\mathcal{R}_U(r')$
    and $v \leadsto r'$. Since $U$ is a forest it follows by the second property
    of a forest that $r = r'$ hence the representative is unique.
\end{proof}

\begin{definition}[Tree Set]
    The set of vertices of the same tree $\mathcal{E}_F(v)$ in a
    forest $F = (V, E)$ is defined as:
    \begin{align*}
        \mathcal{E}_F(v) := \{u : u \in V \land \rho_F(u) = \rho_F(v))\}
    \end{align*}
\end{definition}

\begin{definition}[Partition]\label{def:partition} The
    The set $P \subseteq \mathbb{P}(S)$ is a partition of a set $S$ if:
    \begin{enumerate}
        \item $a \neq \emptyset$ for all $a \in P$
        \item $a \cap b = \emptyset$ for all $a, b \in P$ where $a
        \neq b$
        \item $\bigcup_{a \in P} a = S$
    \end{enumerate}
\end{definition}

\begin{proposition}[Forest Partition]\label{prop:forest-partition}
    A forest $F = (V, E)$ is a partition of $V$ for the following set:
    \begin{align*}
        \{\mathcal{E}_F(v) : v \in V\}
    \end{align*}
\end{proposition}

\begin{proof}
    Let $F = (V, E)$ be a forest. We will show that the set in the proposition
    is a partition of $V$ by showing that it satisfies the three properties in
    definition \ref{def:partition}.
    \begin{enumerate}
        \item By definition of $\mathcal{E}_F(v)$ it can not be empty since for
        $\mathcal{E}_F(v)$ then $\rho_F(v) = \rho_F(v)$. Hence $\mathcal{E}_F(v)
        \neq \emptyset$ for all $v \in V$.
        \item Let $a$ and $b$ be two arbitrary elements in the set such that $a
        \neq b$. By definition of $a$ and $b$ there exists $v_1, v_2 \in V$ such
        that $a = \{u : u \in V \land \rho_F(u) = \rho_F(v_1)\}$ and $b = \{u :
        u \in V \land \rho_F(u) = \rho_F(v_2)\}$. Since $a \neq b$ it follows
        that $\rho_F(v_1) \neq \rho_F(v_2)$ since otherwise $a = b$, hence $a
        \cap b = \emptyset$.
        \item Let $v$ be an arbitrary element in $V$. By proposition
        \ref{prop:roots-path-exist} there exists a root $r \in V$ such that
        $\mathcal{R}_F(r)$ and $v \leadsto r$. By definition of the
        representative it follows that $\rho_F(v) = r$. Now let $a = \{u : u \in
        V \land \rho_F(u) = \rho_F(v)\}$. By definition of $a$ it follows that
        $v \in a$. Since $v$ was arbitrary it follows that $\bigcup_{a \in P} a
        = V$.
    \end{enumerate}
\end{proof}

\begin{definition}[Same Tree Relation]
    The relation $\sim$ on a forest $F$ is defined as:
    \begin{align*}
        u \sim v :\iff u \in \mathcal{E}_F(v)
    \end{align*}
\end{definition}

\begin{corollary}[Same Tree Relation is an Equivalence Relation]
    The relation $\sim$ on a forest $F$ is an equivalence relation due to
    $\{\mathcal{E}_F(v) : v \in V\}$ being a partition of $V$. by proposition
    \ref{prop:forest-partition}.
\end{corollary}

\begin{definition}[Forests with Equivalent Tree Sets]
    Two forests $F = (V, E)$ and $F' = (V', E')$ have equivalent tree sets $F
    \cong F'$ if $V = V$ and
    \begin{itemize}
        \item Vertices are the same $V = V'$.
        \item The tree sets are equivalent $\mathcal{E}_{F}(v) =
        \mathcal{E'}_F(v)$ for all $v \in V$.
    \end{itemize}
\end{definition}

\begin{definition}[Change Parent]
    The function for changing a parent of an element $i$ to $p$ in a forest
    $F = (V, E)$ is defined as:
    \begin{align*}
        \mathcal{S}_F(i, p) := \begin{cases}
            (V, (E \backslash \{(i, p')\}) \cup \{(i, p)\}) & \text{if } (i, p') \in E \\
            (V, E \cup \{(i, p)\}) & \text{if } (i, p') \notin E
        \end{cases}
    \end{align*}
\end{definition}

\begin{definition}[Tree Union]
    The tree union of two elements $v$ and $u$ for a forest $F = (V, E)$ is such
    that $v \sim u$ in a new forest $F' = (V', E')$ and $F'$ satify the
    following properties:
    \begin{enumerate}
        \item $\mathcal{E}_{F'}(v) = \mathcal{E}_{F'}(u) = \mathcal{E}_F(v) \cup \mathcal{E}_F(u)$ and
        \item $\mathcal{E}_{F'}(w) = \mathcal{E}_F(w) \text{ for all } w \in V
        \backslash (\mathcal{E}_F(v) \cup \mathcal{E}_F(u))$.
    \end{enumerate}
\end{definition}

\begin{proposition}[Tree Union by Parent]\label{def:sequential-unions}
    Let forest $F = (V, E)$, $p = \rho_F(u)$ be the representative of $u$ and
    let $q = \rho_F(v)$ be the representative of $v$. Then defined $F'$ as:
    \begin{align*}
        F' &:= \begin{cases}
            F & \text{if } p = q \\
            \mathcal{S}_{F}(q, p) & \text{if } p \neq q
        \end{cases}
    \end{align*}
    Then $u \sim v$ in $F'$ and $F'$ will satisfy the properties of a tree
    union.
\end{proposition}

\begin{proof}
    Let $F = (V, E)$, $p = \rho_F(u)$ be the representative of $u$ and
    let $q = \rho_F(v)$ be the representative of $v$.
    \begin{itemize}
        \item If $p = q$ hence we already have $u \sim v$ in $F$ and since the
        tree sets are equivalent $F' \cong F$ then the properties are satisfied.
        \item If $p \neq q$ then by definition of changing parent then $q$ will
        have parent $p$ in $F'$ and since $q$ is a root it has no parent and $F'
        = (V, E \cup \{(q, p)\})$. Now for all $w \in \mathcal{E}_F(q)$ it holds
        that $w \leadsto q$ or $q = w$ and since $q \leadsto p$ it follows that
        $w \leadsto p$. Hence $w \in \mathcal{E}_{F'}(p)$ for all $w \in
        \mathcal{E}_F(q)$ and trivially $w \in \mathcal{E}_{F'}(p)$ for all $w
        \in \mathcal{E}_F(p)$ so it follows that $\mathcal{E}_{F'}(v) =
        \mathcal{E}_{F'}(u) = \mathcal{E}_F(v) \cup \mathcal{E}_F(u)$. Now let
        $w \in V \backslash (\mathcal{E}_F(v) \cup \mathcal{E}_F(u))$ be an
        arbitrary element. Since $w \centernot\leadsto p$ and $w \centernot\leadsto
        q$ it follows that $w$ has the same parent in $F'$ as in $F$ hence
        $\mathcal{E}_{F'}(w) = \mathcal{E}_F(w)$. 
    \end{itemize}
\end{proof}

\begin{definition}[Union-Find Structure]
    The union-find structure $U$ is a forest $U = (V, E)$ where the vertices $V
    = S$ for some set of elements $S$. The edges $E \subseteq V \times V$
    represent parent relations between elements in $S$ such that $(u, v) \in E$
    means that $u$ has parent $v$.
\end{definition}

\end{document}
